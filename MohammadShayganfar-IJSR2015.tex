%%%%%%%%%%%%%%%%%%%%%%% file template.tex %%%%%%%%%%%%%%%%%%%%%%%%%
%
% This is a general template file for the LaTeX package SVJour3
% for Springer journals.          Springer Heidelberg 2010/09/16
%
% Copy it to a new file with a new name and use it as the basis
% for your article. Delete % signs as needed.
%
% This template includes a few options for different layouts and
% content for various journals. Please consult a previous issue of
% your journal as needed.
%
%%%%%%%%%%%%%%%%%%%%%%%%%%%%%%%%%%%%%%%%%%%%%%%%%%%%%%%%%%%%%%%%%%%
%
% First comes an example EPS file -- just ignore it and
% proceed on the \documentclass line
% your LaTeX will extract the file if required
\begin{filecontents*}{example.eps}
%!PS-Adobe-3.0 EPSF-3.0
%%BoundingBox: 19 19 221 221
%%CreationDate: Mon Sep 29 1997
%%Creator: programmed by hand (JK)
%%EndComments
gsave
newpath
  20 20 moveto
  20 220 lineto
  220 220 lineto
  220 20 lineto
closepath
2 setlinewidth
gsave
  .4 setgray fill
grestore
stroke
grestore
\end{filecontents*}
%
\RequirePackage{fix-cm}
%
%\documentclass{svjour3}                     % onecolumn (standard format)
%\documentclass[smallcondensed]{svjour3}     % onecolumn (ditto)
\documentclass[smallextended]{svjour3}       % onecolumn (second format)
%\documentclass[twocolumn]{svjour3}          % twocolumn
%
\smartqed  % flush right qed marks, e.g. at end of proof
%
\usepackage{graphicx}
%
% \usepackage{mathptmx}      % use Times fonts if available on your TeX system
%
% insert here the call for the packages your document requires
%\usepackage{latexsym}
% etc.
%
% please place your own definitions here and don't use \def but
% \newcommand{}{}
%
% Insert the name of "your journal" with
% \journalname{myjournal}
%
\begin{document}

\title{Emotion-Awareness Improves Human-Robot Collaboration}%\thanks{Grants or
% other notes about the article that should go on the front page should be
%placed here. General acknowledgments should be placed at the end of the article.}
%}

%\subtitle{Do you have a subtitle?\\ If so, write it here}

%\titlerunning{Short form of title}        % if too long for running head

\author{Mohammad Shayganfar \and
        Charles Rich \and
        Candace L. Sidner
}

%\authorrunning{Short form of author list} % if too long for running head

\institute{Mohammad Shayganfar \and Charles Rich \and Candace L. Sidner \at
              100 Institute Road, Worcester, MA, USA 01609-2280 \\
              Tel.: +1 508-831-5357\\
              Fax: +1 508-831-5776\\
              \email{mshayganfar@wpi.edu}\\
              \email{rich@wpi.edu}\\
              \email{sidner@wpi.edu}\\
%             \emph{Present address:} of F. Author  %  if needed
}

\date{Received: date / Accepted: date}
% The correct dates will be entered by the editor


\maketitle

\begin{abstract}\ldots


\keywords{Human-Robot/Agent Collaboration \and Emotion-Awareness \and Affective
Motivational Collaboration Theory}
% \PACS{PACS code1 \and PACS code2 \and more}
% \subclass{MSC code1 \and MSC code2 \and more}
\end{abstract}

\section{Introduction}
\label{intro}

\ldots

\section{Related Work}

\subsection{Emotions in Social Context}

\subsection{Social Functions of Emotions}

\subsection{Affect and Motives}

\subsection{Collaboration Theory}

\section{Example Scenario}
\label{sec:1}
%Text with citations \cite{RefB} and \cite{RefJ}.

\subsection{The Backstory}

The scenario transpires in a NASA's research center. Light, temperature and
other environmental factors are simulated based on conditions on the surface of
the moon. The mission is to finish installing the required solar panels to
provide energy for the operation of NASA's science lab on the moon. Ninety
percent of these panels have already been installed. However, the operation is
now faced with low batteries which forces everyone to be cautious about
consuming energy. The astronaut is inspecting the working conditions in the
field and planning the installation of the remaining panels in collaboration
with the robot. He determines that the sun will cast shadows over the
installation structure, leading to potential difficulties. The astronaut asks
control base to go through the final checks of the robot and prepare it for the
operation.

\subsection{Astronaut-Robot Interaction}

The robot and the astronaut will collaborate with each other to achieve their
shared goal, which is to install two solar panels. They will face various
difficulties, ranging from the task being unpleasant and challenging to
conflicts of their private and/or shared goals occurring because of a blocked or
a protracted sub-task. The robot and the astronaut will go through a series of
assessment processes to figure out a) how did the current blocking happen? b)
why is the current task is blocked? and c) what is the next action they are
going to take? The robot uses its cognitive abilities and its communication
skills to overcome these problems and to motivate the astronaut to propose
alternative tasks. The following is part of an interaction between the astronaut
and the robot during their collaboration on installing solar panels.

\subsection{Agreeing on Shared Goal (Emotion-Awareness)}

\small{
\begin{description}
  \item \textit{\textbf{A1. Astronaut:}} Oh no! Finishing the quality check of
  our installation with this measurement problem is so frustrating. I think we
  should stop now!

  [\textit{Astronaut is frustrated.}]\\

  \item \textit{\textbf{A2. Robot:}} I see. This is frustrating. But, I can help
  you with the measurement tool and we can finish the task as originally
  planned.

  [\textit{Robot perceives Astronaut's frustration and acknowledges that.}]\\
  
  \item \textit{\textbf{A3. Astronaut:}} Can you fix the measurement tool?

  [\textit{Astronaut's emotion is neutral.}]\\
  
  \item \textit{\textbf{A4. Robot:}} The next task is fixing the panel and it
  needs you to prepare and attach the welding rod to your welding tool. To save
  our time, I will fetch another measurement tool while you are preparing your
  welding tool.

  [\textit{Robot perceives Astronaut's neutral emotion, and tries to negotiate
  and provide a fair offer.}]\\

  \item \textit{\textbf{A5. Astronaut:}} That would be great!
  
  [\textit{Astronaut is content.}]
  
\end{description}
}

\subsection{Agreeing on Shared Goal (Emotion-Ignorance)}

\small{
\begin{description}
  \item \textit{\textbf{B1. Astronaut:}} Oh no! Finishing the quality check of
  our installation with this measurement problem is so frustrating. I think we
  should stop now!

  [\textit{Astronaut is frustrated.}]\\

  \item \textit{\textbf{B2. Robot:}} I can help you with the measurement tool,
  or we can terminate this task. What do you want me to do?

  [\textit{Robot does not perceive Astronaut's frustration.}]\\
  
  \item \textit{\textbf{B3. Astronaut:}} As I said the measurement tool does not
  work properly. We can not continue!

  [\textit{Astronaut is frustrated.}]\\
  
  \item \textit{\textbf{B4. Robot:}} TOkay. Do you want me to fix this problem
  or terminate the task?

  [\textit{Robot does not perceive Astronaut's frustration.}]\\

  \item \textit{\textbf{B5. Astronaut:}} Can you fix my measurement tool?
  
  [\textit{Astronaut is frustrated, even more.}]\\
  
  \item \textit{\textbf{B6. Robot:}} I cannot fix your measurement tool, but I
  can fetch another one for you if you want?
  
  [\textit{Despite Astronaut's strong frustration, Robot tries to negotiate.}]\\
  
  \item \textit{\textbf{B7. Astronaut:}} No, I don't want another measurement tool!
  We don't have time for that!
  
  [\textit{Astronaut is angry.}]\\
  
  \item \textit{\textbf{B8. Robot:}} Okay. You want me to terminate this task.
  Terminating this task can influence the quality of installing this solar panel
  which can cause the mission to fail. Or, do you want us to work on another
  task? This can help us to install the panel using your welding tool, but I do
  not know whether the quality of our installation will be acceptable.
  
  [\textit{Not only the Robot does not perceive Astronaut's anger, but also
  continues to negotiate the next step based on the shared plan to select proper
  action.}]\\
  
  \item \textit{\textbf{B9. Astronaut:}} I told you we have this problem and we
  should terminate the mission! We cannot continue without the measurement tool!
  
  [\textit{Astronaut is angry.}]\\
  
\end{description}
}

\section{Affective Motivational Collaboration Theory}

\section{Computational Framework}

\section{Walk Through Computational Examples}

\subsection{Agreeing on Shared Goal (Emotion-Awareness)}

\subsection{Agreeing on Shared Goal (Emotion-Ignorance)}

\subsection{Delegation of a Task (Emotion-Awareness)}

\subsection{Delegation of a Task (Emotion-Ignorance)}

\section{Conclusion and Future Work}

%\label{sec:2} ~\ref{sec:1}

%\paragraph{Paragraph headings} 


% For one-column wide figures use
%\begin{figure}
% Use the relevant command to insert your figure file.
% For example, with the graphicx package use
%  \includegraphics{example.eps}
% figure caption is below the figure
%\caption{Please write your figure caption here}
%\label{fig:1}       % Give a unique label
%\end{figure}
%
% For two-column wide figures use
%\begin{figure*}
% Use the relevant command to insert your figure file.
% For example, with the graphicx package use
%  \includegraphics[width=0.75\textwidth]{example.eps}
% figure caption is below the figure
%\caption{Please write your figure caption here}
%\label{fig:2}       % Give a unique label
%\end{figure*}
%



% For tables use
%\begin{table}
% table caption is above the table
%\caption{Please write your table caption here}
%\label{tab:1}       % Give a unique label
% For LaTeX tables use
%\begin{tabular}{lll}
%\hline\noalign{\smallskip}
%first & second & third  \\
%\noalign{\smallskip}\hline\noalign{\smallskip}
%number & number & number \\
%number & number & number \\
%\noalign{\smallskip}\hline
%\end{tabular}
%\end{table}


%\begin{acknowledgements}
%If you'd like to thank anyone, place your comments here
%and remove the percent signs.
%\end{acknowledgements}

% BibTeX users please use one of
%\bibliographystyle{spbasic}      % basic style, author-year citations
%\bibliographystyle{spmpsci}      % mathematics and physical sciences
%\bibliographystyle{spphys}       % APS-like style for physics
%\bibliography{}   % name your BibTeX data base

% Non-BibTeX users please use
\begin{thebibliography}{}
%
% and use \bibitem to create references. Consult the Instructions
% for authors for reference list style.
%
\bibitem{RefJ}
% Format for Journal Reference
Author, Article title, Journal, Volume, page numbers (year)
% Format for books
\bibitem{RefB}
Author, Book title, page numbers. Publisher, place (year)
% etc
\end{thebibliography}

\end{document}
% end of file template.tex

